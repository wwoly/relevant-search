%%%%%%%%%%%%%%%%%%%%%%%%%%%%%%%%%%%%%%%%%%%%%%%%%%%%%%%%%%%
%% Congratulations, you've made an excellent choice
%% of writing your Tampere University thesis using
%% the LaTeX system. This document attempts to be
%% as complete a template as possible to let you focus
%% on the most important part: the writing itself.
%% Thus the details regarding the visual appearance
%% and even structure have already been worked out
%% for you!
%%
%% I sincerely hope you will find this template useful
%% in completing your thesis project. I've tried to
%% add comments (followed by the % sign) to clarify
%% the structure and purpose of some of the commands.
%% Most of the magic happens in the file tauthesis.cls,
%% which you are more than welcome to take a look at.
%% Just refrain from editing it in the most crucial
%% versions of the thesis!
%%
%% I wish you and your thesis project the best of luck!
%% If this template causes you trouble along the way
%% or if you've any suggestions for improving it,
%% please contact me via email at
%% 
%% ville.koljonen (at) tuni.fi.
%%
%% Yours,
%%
%% Ville Koljonen
%% 16th May 2019
%%
%% PS. This template or its associated class file don't
%% come with a warranty. The content is provided as is,
%% without even the implied promise of fitness to the
%% mentioned purpose. You, as the author of the thesis,
%% are responsible for the entire work, including the
%% provided material. No one else is liable to you for
%% any damage inflicted on you or your thesis, were it
%% caused by using this template or not.
%%%%%%%%%%%%%%%%%%%%%%%%%%%%%%%%%%%%%%%%%%%%%%%%%%%%%%%%%%%

%%%%% NOTICE %%%%%
%% Please read through the entire template
%% (files under ./tex) to find all instructions.
%% It is possible that the attached pdf files
%% do not include the latest information.
%%%%%%%%%%%%%%%%%%

%%%%% INSTRUCTIONS FOR COMPILING THE DOCUMENT %%%%%
%% Overleaf: just click Recompile.
%% Terminal:
%%  1. pdflatex main.tex
%%  2. makeindex -s main.ist -t main.glg -o main.gls main.glo
%%  3. biber main
%%  4. pdflatex main.tex
%%  5. pdflatex main.tex
%% Similar sequence of commands is also required
%% in LaTeX specific editors.
%%%%%%%%%%%%%%%%%%%%%%%%%%%%%%%%%%%%%%%%%%%%%%%%%%%

%%%%% PREAMBLE %%%%%

%%%%% Document class declaration.
% The possible optional arguments are
%   finnish - thesis in Finnish (default)
%   english - thesis in English
%   numeric - citations in numeric style (default)
%   authoryear - citations in author-year style
% Example: \documentclass[english, authoryear]{tauthesis}
%          thesis in English with author-year citations
\documentclass[english, numeric]{doc/tauthesis}

% The glossaries package throws a warning:
% No language module detected for 'finnish'.
% You can safely ignore this. All other
% warnings should be taken care of!

%%%%% Your packages.
% Before adding packages, see if they can be found
% in tauthesis.cls already. If you're not sure that
% you need a certain package, don't include it in
% the document! This can dramatically reduce
% compilation time.

% Graphs
\usepackage{pgfplots}
\pgfplotsset{compat=1.15}

% Subfigures and wrapping text
\usepackage{subcaption}

% Mathematics packages
\usepackage{amsmath, amssymb, amsthm}
%\usepackage{bm}

% Chemistry packages
% Newest mhchem is attached for compatibility
\usepackage{chemfig}
\usepackage[version=4]{mhchem}

%%%%% Your commands.

% Print verbatim LaTeX commands
\newcommand{\verbcommand}[1]{\texttt{\textbackslash #1}}

% Basic theorems in Finnish and in English.
% Remove [chapter] if you wish a simply
% running enumeration.
\newtheorem{lause}{Lause}[chapter]
\newtheorem{theorem}[lause]{Theorem}

% Use the commented version for individually
% enumerated lemmas
\newtheorem{apulause}[lause]{Apulause}
\newtheorem{lemma}[lause]{Lemma}
% \newtheorem{apulause}{Apulause}[chapter]
% \newtheorem{lemma}{Lemma}[chapter]

% Definition style
\theoremstyle{definition}
\newtheorem{maaritelma}{Määritelmä}[chapter]
\newtheorem{definition}[maaritelma]{Definition}
% examples in this style

%%%%% Glossary information.

\loadglsentries[main]{text/abbreviations.tex}
\makeglossaries

%%%%% Citation information.

\addbibresource{references.bib}

\hypersetup{hidelinks}

\begin{document}

%%%%% FRONT MATTER %%%%%

\frontmatter

%%%%% Thesis information and title page.

% The titles of the work. If there is no subtitle,
% leave the arguments empty. Pass the title in
% the primary language as the first argument
% and its translation to the secondary language
% as the second.
\title{Increasing e-commerce conversion rate with 
relevant search results using Elasticsearch}
{Increasing e-commerce conversion rate with 
relevant search results using Elasticsearch}
\subtitle{}

% The author name.
\author{Toni Linnusmäki}

% The examiner information.
% If your work has multiple examiners, replace with
% \examiner[<label>]{<name> \\ <name>}
% where <label> is an appropriate (plural) label,
% e.g. Examiners or Tarkastajat, and <name>s are
% replaced by the examiner names, each on their
% separate line.
\examiner{Prof. Martti Juhola}

% The finishing date of the thesis (YYYY-MM-DD).
\finishdate{2020}{04}{26}

% The type of the thesis (e.g. Kandidaatintyö
% or Master of Science Thesis) in the primary
% and the secondary languages of the thesis.
\thesistype{Master's thesis}{Maisterin tutkielma}

% The faculty and degree programme names in
% the primary and the secondary languages of
% the thesis.
\facultyname
{Faculty of Information Technology and Communication Sciences}
{Informaatioteknologian ja viestinnän tiedekunta}
\programmename
{Master's Programme in Computer Science}
{Tietojenkäsittelyopin maisteriopinnot}

% The keywords to the thesis in the primary and
% the secondary languages of the thesis
\keywords%
    {relevant search, e-commerce, Elasticsearch}
    {hakutulosten relevanssi, verkkokauppa, Elasticsearch}
    

\maketitle


%%%%% Abstracts and preface.

% Write the abstract(s) and the preface
% into a separate file for the sake of clarity.
% Pass the appropriate file name as the first
% argument to these commands. Put the \abstract
% in the primary language first and the
% \otherabstract in the secondary language second.
% Those who do not speak Finnish only need the
% first abstract. The second argument of
% the \preface command takes the place where
% the thesis was signed in.
\abstract{text/abstract.tex}
\otherabstract{text/tiivistelma.tex}
\preface{text/preface.tex}{Tampere}

%%%%% Table of contents.

\tableofcontents

%%%%% Lists of figures, tables, listings and terms.

% Print the lists of figures and/or tables.
% (Un)comment either of these commands as required.
% Both are optional, but if there are many important
% figures/tables, listing them may be a good idea.

% \listoffigures
% \listoftables
% \lstlistoflistings

% % Print the glossary of terms.

% \glossary

%%%%% MAIN MATTER %%%%%

\mainmatter

% Write each of the chapters of the thesis
% into a separate file for the sake of clarity.
% They can be \input as shown below. Give both
% the chapters and their files as descriptive
% names as possible.
\chapter{Introduction}
\label{ch:introduction}

Search engines have been a significant element of the internet for a long time, 
and some of the most visited websites in the world are search engine interfaces, 
like Google and Baidu \cite{alexa}.
By visiting different websites, people might interact with dozens of search engines, 
for example, while writing text to the address bar of a browser or looking for information about a topic of interest.
In addition to interacting with search engines, people might not be aware that they are utilizing one of the most 
potent tools of parsing and matching text.


The history of search engines goes back decades, but the era of modern search engines can be thought to have begun
when \citeauthor{googleInit} \cite{googleInit} introduced their search engine, Google. 
In addition, they proposed methods to index a vast number of documents, in their case web pages, 
and to use a ranking algorithm, PageRank, to rank the documents \cite{googleInit}. 
The usage of indexing and ranking algorithms has been the foundation for search engines ever since.
While using ranking algorithms and the ranking search results by relevancy was not a new idea, 
it was revolutionary to utilize the PageRank ranking algorithm in search engines to provide 
more relevant search results to the queries of the users.


Besides handing most of the navigation on the internet, 
searching is a crucial part of an e-commerce platform. 
\citeauthor{amazonJoyRanking} \cite{amazonJoyRanking} introduced the usage of a search application 
in the largest e-commerce in the world, Amazon, 
and how the search application powers an extensive amount of sales of Amazon.
\citeauthor{relevantSearch} \cite{relevantSearch} introduced text analysis and result ranking methods
that can be efficiently utilized in a search engine to provide relevant search results to users.


While different relevancy methods and the relevancy of the search results has been investigated extensively in the past, 
little research has included the use of the relevancy methods in practice in search applications 
with real-world customers in e-commerce. 
The research has tended to focus on applying the methods in theory to a predetermined dataset to get measurable results
as opposed to focusing on implementing the methods and measuring the performance in practice. 


The primary focus of this thesis was to investigate how the relevancy of the search results affected 
the conversion rate of an e-commerce platform by implementing a selection of methods based on the literature. 
The conversion rate measures what proportion of customers who visited a website bought something, 
and it provided a generic measurement of how the implemented methods performed in 
providing relevant search results to customers.
However, the relevancy of the search result is subjective 
since not all customers are searching for the same thing making the relevancy of a search result 
different for each customer.


The implementation of the methods in this thesis was done to an existing e-commerce platform, 
which serves tens of thousands of customers daily across the Nordic countries. 
Furthermore, a set of tests was performed in the platform to a set of customers by offering different search results 
and measuring how the implementation affected the conversion rate of the e-commerce platform.
Furthermore, by making sure the set of customers was large enough and randomized, 
the relevancy of a different implementation could be compared against a baseline.


The significance of a search application to the conversion rate of the e-commerce platform is 
approximately 2.5x increase in conversion rate
when customers interact with the search functionality on the website compared to
customers who do not interact with the search while visiting the website \cite{powerAnalytics}.
On the other hand, the increase shows that a link exists between showing the search results 
shown to customers and the conversion rate of the e-commerce, 
on the other hand, the increase does not measure the relevancy of the search results.

So, depending on the size of the e-commerce organization, the increase in conversion rate can lead to 
a significant amount of revenue for the organization when customers use their search.
Furthermore, in this thesis, the goal was to increase the conversion rate of the proportion of customers 
using the search application while visiting the website. 
Therefore, how the proportion of customers using the search can be increased was out of the scope of this thesis.


The current search application of the e-commerce platform uses Elasticsearch, which was
the platform that the methods used in this thesis were implemented on.
Elasticsearch is a distributed search and analytics engine that provides 
real-time search and text analysis tools in addition to document storage and indexing capabilities \cite{elasticIntro}. 
\citeauthor{relevantSearch} \cite{relevantSearch} introduced how to build 
a production scale search engine with Elasticsearch in their book. 
While building a large-scale search engine is out of the scope of this study, 
a selection of the text analyzing and search result ranking methods \citeauthor{relevantSearch} \cite{relevantSearch} 
proposed were utilized in this thesis.
Therefore, understanding how the search engine performs in large scale search applications 
is crucial before implementing the methods from the literature in practice.


The tests for this thesis were done as A/B tests in a production environment to capture the 
effects that the different implementations had on the conversion rate and the click-through rate.
However, tests performed for a few weeks might not have captured the full effect of this thesis 
since more relevant search results might have a lasting effect on the reputation of the organization, 
and the increase of a reputation is hard to measure with these kinds of tests. 
Generally, the more satisfied customers are with their experience on a website, 
the more likely they are to return to revisit the website.


The results showed that while the click-through rate of the search results was increased, it did not
show a consistent increase in the conversion rate. 
Furthermore, indicating that while the relevancy of the search results was increased, the conversion
rate was not affected by the change in relevancy.
While the conversion rate was not consistently increased by the different implementations, they
proved a need for future development and provided a direction for it.







%%%%% Bibliography/references.

% Print the bibliography according to the
% information in ./references.bib and
% the in-line citations used in the body of
% the thesis.
% \emergencystretch=2em
\printbibliography[heading=bibintoc]

%%%%% Appendices.

% Use only if it clarifies the structure of
% the document. Remember to introduce each
% appendix and its content.

% \begin{appendices}

% % \chapter{Esimerkkiliite}
% \label{ch:appendices}
% % \input{./tex/liite.tex}

% \end{appendices}

\end{document}