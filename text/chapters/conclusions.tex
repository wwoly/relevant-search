


The purpose of this chapter is to summarize the goals of this thesis and identify some problems discovered
during the tests.
The chapter summarizes the research goal of this thesis in addition to the additional conclusions drawn
from the results.
The conclusions each include the subjects that could be researched further to improve the solution proposed in this thesis
if some problems were found with the chosen implementation.


\section{The effect on conversion rate}

The primary purpose of this thesis was to increase the conversion rate of an existing e-commerce platform by offering
more relevant search results to users. 
However, while the concluded tests did not indicate a significant increase in the conversion rate, they still
indicated that there might be room to improve the solution, as the conversion rate still was affected by different
tests.
While there were a lot of useful tools provided by \citeauthor{relevantSearch} \cite{relevantSearch} in their book, 
using them, in practice, in a large scale search engine with multiple languages proved to be a difficult task.

Furthermore, this thesis still proved that the chosen direction was a success, and the development in the future
should focus on the subjects introduced in this thesis.
When it comes to increasing the conversion rate, much more work needs to be put into investigating 
how the conversion rate could be increased in the search result list.
Since customers seemed to find interesting products according to the tests, but still, 
that did not translate into the conversion rate.


\section{Relevance of the search results}

The other main point of this thesis was the relevance of the search results and how it affected the conversion rate 
of the e-commerce platform.
Based on the concluded tests, the conversion rate was not affected significantly by the different implementations.
However, as previously stated, the customers seemed to find interesting products, at least when measured with the 
click-through rate. 
Based on the concluded tests, there seemed to be little correlation with the relevance of the search results
and the conversion rate in the specific e-commerce platform.

While, in literature, the conversion rate was affected by the search results, in reality, the conversion
rate is affected by many variables besides the search results.
Therefore, the results could be explained by 
that the customers may have been searching for information about products and 
not that interested in buying them.
On the one hand, the previous example could be explained with the lack of development the search application has seen, 
which could have affected the behavior of the customers to drive them more towards searching for information.
On the other hand, the explanation could be 
that the products offered by the e-commerce platform do not satisfy the needs of the customer base well.

In addition, during development it was found that the content in the products is not very searchable.
For instance, the example with "pesukone" introduced previously.
Furthermore, meaning that future development could also focus on generating meaningful tokens.
For instance, mining content from the web about the products and using the previous search terms as a base,
the already existing \emph{searchTerms} field could be filled with actually searchable content.

In addition to generating content for the search engine, with properly generating the search token,
the usage of wildcards in search could be made obsolete.
Well generated tokens enable the usage of fuzzy search instead of the wildcard, which again allows
the search engine to match words even though the customer has misspelled it.
Furthermore, it could drastically increase the usability of the search application in the future as there would not 
need to write the search term correctly.


\section{Testing the implementations}

In this thesis, four different tests were concluded over six weeks.
With the four tests, only a handful of different implementations could be tested, meaning that the testing
of different implementation is prolonged.
While the results from A/B tests are reliable, the time used to conclude them is not feasible.


As \citeauthor{relevantSearch} \cite{relevantSearch} said that with search application development, the goal
is to fail fast, meaning that a new implementation should be tested, and if not working as expected,
it should be disregarded. 
Using production A/B tests in a fail-fast development is not possible since the tests need to run for days
to get meaningful results.
In addition, since the data was collected in different periods and from different countries, the tests
are not comparable between each other.


In the future, a testing platform purposefully built for testing search engines could be beneficial 
to run simulated tests to see what is working.
The tests concluded for this thesis could have been done possibly in a matter of hours with a testing platform
that simulates real customers.
While the results might not have been as precise, still instead of waiting for results from the tests,
more development could have been done.
In addition, the simulated platform could provide comparable results across different tests since the data used
could stay more or less the same.





This chapter provided an overview of the conclusions drawn from the results and included subjects that could be
developed in the future.
While the results of the concluded tests were not optimal as the conversion rate was not increased,
much was learned during the development.
Therefore, this thesis proved to be an excellent learning experience in how search engines function at a lower level, 
in addition to how Elasticsearch performs as a production scale search application.


