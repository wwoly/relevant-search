
Search engines have been a significant element of the internet for a long time, 
and some of the most visited websites in the world are search engine interfaces, 
like Google and Baidu \cite{alexa}.
By visiting different websites, people might interact with dozens of search engines, 
for example, while writing text to the address bar of a browser or looking for information about a topic of interest.
In addition to interacting with search engines, people might not be aware that they are utilizing one of the most 
potent tools of parsing and matching text.


The history of search engines goes back decades, but the era of modern search engines can be thought to have begun
when \citeauthor{googleInit} \cite{googleInit} introduced their search engine, Google. 
In addition, they proposed methods to index a vast number of documents, in their case web pages, 
and to use a ranking algorithm, PageRank, to rank the documents \cite{googleInit}. 
While using ranking algorithms and the ranking search results by relevance was not a new idea, 
it was revolutionary to utilize the PageRank ranking algorithm in search engines to provide 
more relevant search results to the queries of the users.


Besides handing most of the navigation on the internet, 
searching is a crucial part of an e-commerce platform. 
\citeauthor{amazonJoyRanking} \cite{amazonJoyRanking} introduced the usage of a search application 
in the largest e-commerce in the world, Amazon, 
and how the search application powers an extensive amount of sales of Amazon.
\citeauthor{relevantSearch} \cite{relevantSearch} introduced text analysis and result ranking methods
that can be efficiently utilized in a search engine to provide relevant search results to users.


While different relevance methods and the relevance of the search results have been investigated extensively in the past, 
research has included the use of the relevance methods in practice in search applications 
with real-world customers in e-commerce to a certain extent. 
The research has tended to focus on applying the methods in theory to a predetermined dataset to get measurable results
as opposed to focusing on implementing the methods and measuring the performance in practice. 



The primary focus of this thesis was to investigate how the relevance of the search results affected 
the conversion rate of an e-commerce platform by implementing some methods based on the literature. 
The conversion rate measures the proportion of customers who visited a website bought something, 
and it provided a generic measurement of how the implemented methods performed in 
providing relevant search results to customers.
However, the relevance of the search result is subjective 
since not all customers are searching for the same thing making the relevance of a search result 
different for each customer.


The implementation of the methods in this thesis was done to an existing e-commerce platform, 
which serves tens of thousands of customers daily across the Nordic countries. 
The tests were performed in the platform to a set of customers by offering different search results 
and measuring how the implementation affected the conversion rate of the e-commerce platform.
Furthermore, by making sure the set of customers was large enough and randomized, 
the relevance of a different implementation could be compared against a baseline.


The impact of a search application to the conversion rate of e-commerce platform is 
approximately 2.5x increase in conversion rate
when customers interact with the search functionality on the website compared to
customers who do not interact with the search while visiting the website \cite{powerAnalytics}.
However, it should be noted that even if a link exists between the search results and the conversion 
rate of e-commerce, it does not necessarily mean that the results are relevant.


The increase in conversion rate when the search is used can lead to 
a significant amount of revenue for the organization, depending on the size of the e-commerce organization.
Additionally, the conversion rate of the e-commerce platform would increase if more customers were steered towards
using the search application.
However, in this thesis, only the increase of the conversion rate using relevant search results was studied.
Some closely related research questions such as
how the proportion of customers using the search can be increased are out of the scope of this thesis.


The current search application of the e-commerce platform uses Elasticsearch, which is
the platform that the methods used in this thesis were implemented on.
Elasticsearch is a large-scale distributed search and analytics engine that provides 
real-time search and text analysis tools in addition to document storage and indexing capabilities 
\cite{elasticIntro, relevantSearch}.
While building a large-scale search engine is out of the scope of this study, 
a selection of text analyzing tools and search result ranking methods proposed by 
\citeauthor{relevantSearch} \cite{relevantSearch} were utilized in this thesis.
Therefore, understanding how the search engine performs in large scale search applications 
is crucial before implementing the methods from the literature in practice.


The tests for this thesis were done in a production environment to capture the 
effects that the different implementations had on the conversion rate and the click-through rate.
However, tests performed for a few weeks might not have captured the full effect of this thesis 
since more relevant search results might have a lasting effect on the reputation of the organization, 
which is hard to measure with these kinds of tests. 
Generally, the more satisfied customers are with their experience on a website, 
the more likely they are to return to revisit the website.


The results showed that while the click-through rate of the search results was increased, there was no
consistent increase in the conversion rate, and, while the relevance of the search results was increased, the conversion
rate was not affected by the change in relevance.
Although the conversion rate was not consistently increased by the different implementations, they
proved a need for future development and provided a direction for it.



