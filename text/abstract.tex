
% ---------------------------------------
% Abstract and keywords
% ---------------------------------------

\noindent 
Search engines have grown vastly and are now part of our everyday life 
by providing more straightforward navigation on the internet.
Besides, search engines are an integrated part of many websites, including e-commerce 
platforms.
In this thesis, the focus was to increase the relevance of the search results
of an existing e-commerce platform and analyze how the conversion rate is affected.
Furthermore, the conversion rate describes the percentage of visitors that 
bought something from the total number of visitors on a website, and it is a widely
used measurement in e-commerce.


Previous research has focused mostly on studying different ranking algorithms, but not
many are included in the existing search engine frameworks.
For this thesis, the purpose was to investigate how the methods from previous 
research could be utilized in an e-commerce platform with real customers.
Furthermore, the conducted tests were over six weeks and included tens of thousands
of customers.

The data that was collected during the testing phases of the proposed methods
shows how the changes affect the conversion rate of the e-commerce platform.
In addition to the conversion rate, the click-through rate was analyzed since 
it seemed to capture the changes better.
The results include both measurements and analysis of how the different
parts possibly affected the measurements.

While the results did not indicate that the conversion rate of the e-commerce platform
was increased by the proposed methods, the results showed that with a simple 
solution, the existing process could be replaced.
In addition, the results showed the need and direction for future development, which could
have a further effect on the conversion rate of an e-commerce platform.


~

\noindent\textbf{Keywords:} search engine, relevant search results, e-commerce, Elasticsearch.

~

\noindent The originality of this thesis has been checked using the Turnitin Originality Check service.

\clearpage
